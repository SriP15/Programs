\documentclass[11pt]{article}
\usepackage{fullpage}

\title{Assignment 1: \emph{Design}}
\author{Sriramya Prayaga}
\date{\today}

\begin{document}\maketitle

This assignment asks us to write a bash script that creates various 
graphs with Gnuplot using data obtained from the Collatz sequence 
(provided by the given file collatz.c). The first plot graphs the length
of the Collatz sequences of the numbers 2 to 10,000. The second graph 
plots the maximum value of the Collatz sequences of the numbers
2 to 10,000. The third graph reveals the frequency of each Collatz 
sequence length for the numbers from 2 to 10,000. 

While doing this assignment, as I became interested in seeing what the lengths
of the sequences would be if I ran the given collatz program randomly, I also 
plotted the value of 99 random runs against their lengths using dots.

\section{Pseudocode}

(graph 1 pseudocode below)

if the data file for this graph already exists, delete it

for each i in the range of 2 to 10,000, print i and run the collatz.c file 
(using i as the input to the program) to count the number of output lines 
each i creates (length of sequence) and append i and this value to a data file.


Then with Gnuplot:


      \begin{itemize}
       \item set the terminal to pdf format
       \item set the output to the desired pdf file
       \item set the title of the graph to "Collatz Sequence Lengths"
       \item set the label of the x-axis to "x"
       \item set the label of the y-axis to "length"
       \item set the key off
       \item plot the values in the data file with dots
     \end{itemize}


(graph 2 pseudocode below)

if the data file for this graph already exists, delete it

for each i in range 2 to 10,000, print i and run the collatz.c file
(using i as the input), then reversely sort the numerical output values of
each i to append i and its first sorted value to a data file.


Then with Gnuplot:

      
      \begin{itemize}
       \item set the terminal to pdf format
       \item set the output to the desired pdf file
       \item set the title of the graph to "Maximum Collatz Sequence Value"
       \item set the label of the x-axis to "n"        
       \item set the label of the y-axis to "value"
       \item set the key off
       \item plot the values in the data file using the default x-range and
	     specified y-range [0:100000] with dots
      \end{itemize}


(graph 3 pseudocode below)

if the data file for this graph already exists, delete it

get the values of the second column of the data file of the first graph,
sort those values, compute the frequency of each length, and redirect
these new values to a new data file

Then with Gnuplot:


      \begin{itemize}
       \item set the terminal to pdf format
       \item set the output to the desired pdf file
       \item set the title of the graph to "Maximum Collatz Sequence Value"
       \item set the label of the x-axis to "length"
       \item set the label of the y-axis to "frequency"
       \item set the key off
       \item set xtics values with 0,25,225
       \item plot the values in the latter data file using the default y-range and 
	     specified x-range [0:225] and treating the second column in the latter
	     data file as x, and first column as y with dots
      \end{itemize}


(graph 4 pseudocode below)

if the data file for this graph already exists, delete it

for each i in the range of 2 to 100, print i and run the collatz.c file (randomly) 
to count the number of output lines each i creates (length of the sequence), sleep 
for 1 second, and append i and its corresponding length to a data file.


Then with Gnuplot:


     \begin{itemize}
      \item  set the terminal to pdf format
      \item  set the output to the desired pdf file
      \item  set the title of the graph to "Lengths of 100 Random Collatz Sequences 
	     (with different seeds)"
      \item  set the label of the x-axis to "random sequence index number (max index
	     is 100)"
      \item  set the label of the y-axis to "length"
      \item  set the key off
      \item  plot the values in the data file with dots
     \end{itemize}

\end{Pseudocode}
\section{Explanation of Pseudocode}

The first two plots and the last plot are constructed from data created by
for-loops, which can be observed above in the pseudocode. My code also deletes the
data file for each plot if it already exists before the data is generated for
the graphs. I did this because the for-loops would append more data to the existing
file if I didn't delete the file (this would mess up the plotting.)

\subsection{Explanation of the First Graph}

The first graph reveals the correlation between values of n (from i=2 to i=10,000) 
and the corresponding Collatz sequence length. This code can be found starting from
the first for-loop. It uses pipes to feed the data obtained from running the program
into the wc - l command --it counts how many lines of code are in the output. This 
actually determines the length of any particular Collatz sequence for i because the
collatz.c program writes each value of the sequence on a new line. The values of i
and its lengths are redirected to a data file, where they can then be plotted with
the specifications listed in the pseudocode listed above (for example, dots.)

\end{Explanation of the First Graph}

\subsection{Explanation of the Second Graph}
The second plot shows the values of n with the corresponding maximum value of the
Collatz sequence. The code for this can be found starting from the second for-loop
in the pseudocode (from i=2 to i=10,000), and it also uses pipes. It takes the 
values obtained from running collatz.c (with i as the input), and sorts these numbers
in reverse order using the command sort -nr. This is done by feeding the data into
the command with the usage of pipes. The data is then again fed into another command
(head -n 1) with a pipe to determine the first line of data. I used the command 'head'
because once the numerical data is sorted in reverse order, the first line of the
output would be the greatest value in the sequence. This would allow us to redirect
both the values of i and its corresponding maximum number to a data file. Then this
data can be plotted using Gnuplot with the ranges specified in the pseudocode and
using dots.  

\end{Explanation of the Second Graph}

\subsection{Explanation of the Third Graph}

As I already had the data file created from the first graph (that counts the length of the
sequences), I simply reused it for plotting the third graph. I used awk to extract the
second column of the data file and piped it into the command sort -n. The values in 
this data file are sorted numerically, and the frequency of each value is counted using
the uniq command. Then these newly obtained values --that is, the frequency and the
corresponding lengths-- are appended to a new data file. This latter data file is
the one that is used to make the frequency vs length plot. Once the data is entered
into the second data file, the plot is graphed through Gnuplot with the specifications
given in the pseudocode (draw using filled boxes and with the x and y columns of the
data file switched.)

\end{Explanation of the Third Graph}

\subsection{Explanation of the Fourth Graph}

As mentioned earlier, I also created a fourth graph using much fewer data points.
The code for this can be seen starting with the fourth for-loop. I ran the program
99 times randomly (from i=0 to i=99), and plotted the length of the sequence that 
is corresponded to each i's random run. The code for this is very similar to that
of the first graph, however, because the collatz.c program uses the current time as
the seed, the for-loop would only produce the same numbers each time if I didn't make
additional changes. That is why I wrote the code sleep(1) --to ensure a new seed
is generated for the program. I simply plotted the data with dots using Gnuplot.

\end{Explanation of the Fourth Graph}

\end{Explanation of the Pseudocode}

\end{document}


